\chapter{Using this custom VFS}

\section*{Compilation and setup}

On Debian-based Linux systems (such as Ubuntu), you may install these dependencies with APT:

I will just copy-paste it from the final CI/CD script


\section{Running the VFS using Docker}\label{sec:docker-guide}

The following code block shows the Docker commands used to create and run the custom VFS container.

The first step is optimal, depending whether the container will be used for debugging or actual use, we can create a Docker volume.

\begin{lstlisting}[language=bash, basicstyle=\ttfamily\small]
docker volume create customvfs-test
\end{lstlisting}

Next, we build the Docker image:

\begin{lstlisting}[language=bash, basicstyle=\ttfamily\small]
docker build -t customvfs .
\end{lstlisting}

And after that, we have multiple options for running the container.
Depending on whether we created a Docker volume, we can either mount it or not.
In case of mounting to an actual directory, replace the \texttt{customvfs-test} with the path to the directory.

\begin{lstlisting}[language=bash, basicstyle=\ttfamily\small]
docker run --name customvfs-container --rm -it --privileged \
-v customvfs-test:/mnt/fs customvfs
\end{lstlisting}

We also have the option to run the container in the background.
And that would require diffrent command passed to docker.

\begin{Verbatim}
apt-get install build-essential cmake libfuse-dev crypto++-dev ...
\end{Verbatim}

To unpack and compile the software, proceed as follows:

\begin{Verbatim}
...
cmake ..
\end{Verbatim}

\subsection*{Mount the file system}

Mounting the file system is rather simple.
All you need to do is a directory and to run the following command:

\begin{Verbatim}
  CustomVFS /path/to/mountpoint
\end{Verbatim}

Some option notes

\subsection*{Unmount the file system}

\begin{Verbatim}
  fusermount -u /path/to/mountpoint
\end{Verbatim}

\section*{Section}

You are all set.

Let's just learn how to use it - CLI tools