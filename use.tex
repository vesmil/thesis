\chapter{Usage}

\section*{Mount the VFS}

Mounting the VFS is rather simple, you just need to run the following command:

\begin{lstlisting}[language=bash, basicstyle=\ttfamily\small]
  .\customvfs_exec /path/to/mountpoint
\end{lstlisting}

In case you want to use already existing directory in this VFS, you can use the \texttt{-b} or \texttt{-{}-backing} option with a specified path to the directory.
Note that the directory will be altered and used for helper files, hooks, and other stuff.

Another option you might use is \texttt{-f} or \texttt{-{}-fuse-args} which will pass the following argument to the FUSE library.
In case of multiple arguments, you have to use quotes.

\subsection*{Example}

The following command will mount the file system to \texttt{/path/to/mountpoint} with backing directory set to \texttt{/path/to/backing/directory}.
Besides that, it will also pass \texttt{-o nonempty -f -d} to the FUSE main, meaning it will run in foreground, print debug messages and allow mounting to non-empty directory.

\begin{lstlisting}[language=bash, basicstyle=\ttfamily\small]
  .\customvfs_exec -b /path/to/backing/directory \
                   -f "-o nonempty -f -d" \
                   /path/to/mountpoint
\end{lstlisting}

\subsection*{Unmount}

That is even simpler as you just have to run:

\begin{lstlisting}[language=bash, basicstyle=\ttfamily\small]
  fusermount -u /path/to/mountpoint
  # or in case of macOS - umount /path/to/mountpoint
\end{lstlisting}

\section*{Commands}

Now when you are all set up, you can start using the file system.
All standard operations stay the same as you are used to.
But if you wish to use the custom features, you need to use one of the two provided tools.

\subsection*{Encryption}

The \texttt{cvfs\_encrypt} tool is used for encrypting files or directories.
The following commands can be utilized:

\begin{enumerate}
    \setlength\itemsep{-0.1em}
    \item \textbf{Encrypt a file:} \\
    \begin{BVerbatim}[baseline=t,boxwidth=10cm]
  ./encryption --lock <file>
    \end{BVerbatim}

    \item \textbf{Encrypt a file with a custom key:} \\
    \begin{BVerbatim}[baseline=t,boxwidth=10cm]
  ./encryption --lock <file> --key <key>
    \end{BVerbatim}

    \item \textbf{Decrypt a file:} \\
    \begin{BVerbatim}[baseline=t,boxwidth=10cm]
  ./encryption --unlock <file>
    \end{BVerbatim}

    \item \textbf{Decrypt a file with a custom key:} \\
    \begin{BVerbatim}[baseline=t,boxwidth=10cm]
  ./encryption --unlock <file> --key <key>
    \end{BVerbatim}

    \item \textbf{Generate a new encryption key:} \\
    \begin{BVerbatim}[baseline=t,boxwidth=10cm]
  ./encryption --generate <file>
    \end{BVerbatim}

    \item \textbf{Set a default path for the encryption key:} \\
    \begin{BVerbatim}[baseline=t,boxwidth=10cm]
  ./encryption --set-key-path <file>
    \end{BVerbatim}
\end{enumerate}

\subsection*{Versioning}

The \texttt{cvfs\_version} tool is used for file versioning, and available are the following commands:

\begin{enumerate}
    \setlength\itemsep{-0.1em}
    \item \textbf{List all versions of a file:} \\
    \begin{BVerbatim}[baseline=t,boxwidth=10cm]
  ./versioning --list <key>
    \end{BVerbatim}

    \item \textbf{Restore a file to a specific version:} \\
    \begin{BVerbatim}[baseline=t,boxwidth=10cm]
  ./versioning --restore <version> <key>
    \end{BVerbatim}

    \item \textbf{Delete a specific version of a file:} \\
    \begin{BVerbatim}[baseline=t,boxwidth=10cm]
  ./versioning --delete <version> <key>
    \end{BVerbatim}

    \item \textbf{Delete all versions of a file:} \\
    \begin{BVerbatim}[baseline=t,boxwidth=10cm]
  ./versioning --deleteAll <key>
    \end{BVerbatim}
\end{enumerate}