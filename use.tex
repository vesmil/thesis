\chapter{Usage}

\section*{Mount the file system}

Mounting the file system is rather simple.
All you need to do is a directory and to run the following command:

\begin{lstlisting}[language=bash, basicstyle=\ttfamily\small]
  .\customvfs_exec /path/to/mountpoint
\end{lstlisting}

But there are multiple options you can use.

First of all, mounting point does not have to be a positional argument, and you can use \texttt{-m} or \texttt{-{}-mountpoint} option.

In case you want to use already existing directory in this VFS, you can use the \texttt{-b} or \texttt{-{}-backing} option with a specified path to the directory.
Note that the directory will be altered.

Another option you might use is \texttt{-f} or \texttt{-{}-fuse-args} which will pass the following argument to the FUSE library.
In case of multiple arguments, do not forget to use quotes.

\subsection*{Example}

\begin{lstlisting}[language=bash, basicstyle=\ttfamily\small]
  .\customvfs_exec -b /path/to/backing/directory -m /path/to/mountpoint -f "-o nonempty -f -d"
\end{lstlisting}

This will mount the file system to \texttt{/path/to/mountpoint} with \texttt{/path/to/backing/directory} as backing directory and with \texttt{-o nonempty -f -d} as FUSE arguments.
Meaning it will run in foreground, print debug messages and allow mounting to non-empty directory.

\subsection*{Unmount the file system}

\begin{lstlisting}[language=bash, basicstyle=\ttfamily\small]
  fusermount -u /path/to/mountpoint
\end{lstlisting}

Or in case of macOS:

\begin{lstlisting}[language=bash, basicstyle=\ttfamily\small]
    umount /path/to/mountpoint
\end{lstlisting}

\section*{Commands}

Now when you are all set up, you can start using the file system.
All standard operations stay the same as you are used to.
Well, that was the whole point of this project.

But if you wish to use the custom features, you need to use one of the separate tools.
There are two of them, one for each feature.

\subsection*{Encryption}

To encrypt a file or directory, you need to use the \texttt{cvfs\_encrypt} tool.

\begin{lstlisting}[language=bash, basicstyle=\ttfamily\small]
./encryption --lock <file>
\end{lstlisting}

This will encrypt the given file.
If you wish to provide a custom encryption key, you can do so using the \texttt{--key} flag:

\begin{lstlisting}[language=bash, basicstyle=\ttfamily\small]
./encryption --lock <file> --key <key>
\end{lstlisting}

To decrypt a file, you can use the \texttt{--unlock} flag:

\begin{lstlisting}[language=bash, basicstyle=\ttfamily\small]
./encryption --unlock <file>
\end{lstlisting}

If you encrypted the file with a custom key, you will need to provide it again to decrypt:

\begin{lstlisting}[language=bash, basicstyle=\ttfamily\small]
./encryption --unlock <file> --key <key>
\end{lstlisting}

If you wish to generate a new encryption key, you can use the \texttt{--generate} flag:

\begin{lstlisting}[language=bash, basicstyle=\ttfamily\small]
./encryption --generate <file>
\end{lstlisting}

This will generate a new encryption key and save it to the specified file.
Note that the file needs to be located on the mounted file system.
Even though you may eventually move it.

If you wish to set a default path for the encryption key, you can use the \texttt{--set-key-path} flag:

\begin{lstlisting}[language=bash, basicstyle=\ttfamily\small]
./encryption --set-key-path <file>
\end{lstlisting}

\subsection*{Versioning}

For versioning of files, the \texttt{cvfs\_version} tool is used.

To list all versions of a file, you can use the \texttt{--list} flag:

\begin{lstlisting}[language=bash, basicstyle=\ttfamily\small]
./versioning --list <key>
\end{lstlisting}

To restore a file to a specific version, you can use the \texttt{--restore} flag followed by the version number:

\begin{lstlisting}[language=bash, basicstyle=\ttfamily\small]
./versioning --restore \texttt{<version>} <key>
\end{lstlisting}

If you wish to delete a specific version of a file, you can use the \texttt{--delete} flag followed by the version number:

\begin{lstlisting}[language=bash, basicstyle=\ttfamily\small]
./versioning --delete \texttt{<version>} <key>
\end{lstlisting}

To delete all versions of a file, you can use the \texttt{--deleteAll} flag:

\begin{lstlisting}[language=bash, basicstyle=\ttfamily\small]
./versioning --deleteAll <key>
\end{lstlisting}
