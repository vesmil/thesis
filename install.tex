\chapter{Installation}

\section{Dependencies}\label{sec:dependencies}

It is important to note that this project was originally meant for Linux systems.
Because of that, there might be some issues with other operating systems.

Before we even start, we need to clone the repository.

\begin{lstlisting}[language=bash, basicstyle=\ttfamily\small]
git clone git@gitlab.mff.cuni.cz:teaching/\
theses/yaghob/vesely-milan/source-code.git
\end{lstlisting}

or in case you do not have access to MFF GitLab:

\begin{lstlisting}[language=bash, basicstyle=\ttfamily\small]
git clone https://github.com/vesmil/thesis-source
\end{lstlisting}

After that we need to install the dependencies.

\subsection*{Linux}

On Debian-based Linux systems (such as Ubuntu), you may install these dependencies with apt:

\begin{lstlisting}[language=bash, basicstyle=\ttfamily\small]
sudo apt-get update && sudo apt-get install -y \
     clang cmake make g++ fuse libfuse-dev \
     libgtest-dev libsodium-dev \
     pkg-config libboost-program-options-dev
\end{lstlisting}

Other linux distributions may have different package names for the dependencies.
But overall, it should be similar.

\subsubsection*{Docker}\label{subsubsec:docker-guide}

Even though there is a functioning Docker file, its main purpose was for testing.
The Reason behind that is that FUSE requires kernel module, and Docker would have to reuse the host's kernel.
This, however, does not provide any benefits over using the host directly.

In case of different Linux distributions, however, you can use the Dockerfile in case of problems with dependencies.

\subsection*{macOS}

For macOS, you have to use Homebrew:

\begin{lstlisting}[language=bash, basicstyle=\ttfamily\small]
brew install --cask macfuse && \
brew install libsodium boost googletest cmake llvm
\end{lstlisting}

Even though macs are supported and tested for both Intel and Apple Silicon; the setup does include some extra steps.

Namely, you need to enable support for kernel extensions.
This is done through recovery mode, but I will not go into details as there are multiple guides on the internet on how to do that\cite{macos-kext}.
I may also note that in case you skip this step, the program will prompt you to do so upon running.

Unfortunately, the current implementation of encryption module is not compatible with Apple Silicon.
This is due to the fact that libsodium does not support it yet.
However, the rest of the project should work just fine.

\subsection*{Windows}

The project is not supported on Windows as both WinFSP and Dokany have different API than FUSE\@.
However, it could be easily ported to Windows by providing a different implementation of \texttt{FuseWrapper}.
All the other used libraries should be platform independent.

Nonetheless, it could be run on Windows using WSL\@.

\section{Build}\label{sec:build}

Once the environment is set up, you can build the project easily using CMake.

\begin{lstlisting}[language=bash, basicstyle=\ttfamily\small]
mkdir build
cd build
\end{lstlisting}

\noindent Configure the project using CMake:

\begin{lstlisting}[language=bash, basicstyle=\ttfamily\small]
cmake ..
make
\end{lstlisting}

