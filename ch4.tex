\chapter{Preparation}
\label{chap:preparation}

With the architecture and design decisions established, the focus now shifts to the implementation of the custom and extendable VFS\@.
This concise chapter will outline the tools and technologies employed throughout the development process, ensuring a solid foundation for the implementation phase.


\section{Cross-platform build System}\label{sec:build-system-and-cross-platform-challenges}

CMake\cite{cmake}, an open-source build system, has been selected for this project due to its cross-platform compatibility, user-friendliness, and extensive support for various platforms.
CMake generates native build environments, such as Makefiles or project files for integrated development environments (IDEs), streamlining the development process.

\subsection{Platform-Specific Challenges}\label{subsec:platform-specific-challenges}

Despite the cross-platform advantages provided by CMake, several platform-specific challenges arose during development.
For instance, running applications on an M1 Mac proved difficult due to the lack of compatible binaries for the hardware.
While building the applications from source was an option, there were no compelling reasons to do so for the purposes of this thesis.
Furthermore, the osxfuse port had not received an update in five years, exacerbating the situation.
On Windows, WinFsp\cite{winfsp} is required to ensure FUSE compatibility, introducing additional hurdles for seamless cross-platform development.

Unfortunately, the current implementation of the encryption module is not compatible with Apple Silicon.
This was found out during the final stages of development, as the project was being tested on an M1 Mac.
The reason for this is that libsodium does not properly support Apple Silicon yet.
However, the rest of the project should work just fine, and the part of encryption module could be easily replaced with a different implementation.


\section{Gitlab CI}\label{sec:gitlab-ci}

The MFF faculty GitLab instance was utilized for this project\cite{gitlab_mff}.
GitLab CI/CD pipelines were employed to automate various tasks, such as checking code formatting with clang-format, building the project, running Google Test unit tests, and providing the ability to download build artifacts.
This approach facilitated a more efficient and organized development process.

Regarding the clang-format check, the project was configured to use the Google C++ style guide\cite{google_cpp_style_guide} with a few minor modifications.
The Google style guide was also used regarding the naming convention, although that unfortunately is not enforced by the clang-format tool.

\subsection{Testing}\label{subsec:testing}

For the testing purposes, Google Test was integrated into the GitLab CI/CD pipelines.
This allowed for the execution of simple tests on every commit, such as creating directories and files.
Apart from testing the added VFS functionality, the tests also check helper classes, such as the \texttt{Encryptor} class for encryption and decryption, or a custom library to allow prefixing file names with additional information.

To optimize the testing process, the tests were run on a shared mount-point that was cleared every time, although this approach is not ideal as it can make it harder to determine the root cause of a failure.
Nonetheless, the primary purpose of these tests was to ensure that basic functionality was not broken with each commit, rather than to diagnose the cause of any failures that occurred.

Originally, the tests were supposed to mount and unmount the VFS, but this proved to be problematic.
Debugging the tests was difficult in particular, since the \texttt{gdb} did not properly handle detached processes and would not stop on VFS breakpoints.
As a result of that, mounting of the VFS is now done manually before running the tests and the mount-point needs to be passed as an argument.

Another fact that proved to be problematic was that FUSE requires a kernel module, as mentioned in section~\ref{sec:docker}.
Because of that, the tests needed to run on a custom runner with privileged access, but as it would be a complication to let them run all the time, CI/CD pipeline was limited to only run helper method tests.

In the end, the tests were not as extensive as could be, but they still provided a good baseline for the project.


\section{Docker}\label{sec:docker}

Docker\cite{docker} was also incorporated into the project because of several reasons.
First, it allowed for the creation of a consistent and reproducible development environment, ensuring that the project could be built and run across different platforms without issues.

Additionally, Docker provided two methods for debugging or even demoing the VFS: one that displays debug output in a terminal, and another that runs the VFS in the background.
By integrating Docker into the project, the development and deployment process was simplified, allowing for a more efficient and effective implementation of the custom and extendable VFS\@.

To create the Docker environment for the project, a base Ubuntu 20.04 image was used.
Of course, some dependencies included FUSE, GTest, Boost Program Option, and Libsodium were also installed.
After that, the source code is copied into the container and the project is built.

Sadly, the Docker container in this project has one major drawback, as FUSE requires a kernel module, which Docker is not able to provide.
Meaning that the Docker would still need access to the fuse device, complicating the process of running the container across different platforms.
Besides that, it also needs to run with privileged access, which is not ideal.

Overall, the incorporation of Docker into the project provided a consistent and reproducible development environment, simplified the deployment process, and allowed for easier debugging and testing of the custom VFS\@.
