\chapter{Implementations}

\section{FUSE in C++}

Necessary operations

struct fuse_operations {
    int (getattr) (const char , struct stat );
    int (readlink) (const char , char , size_t);
    int (getdir) (const char , fuse_dirh_t, fuse_dirfil_t);
    int (mknod) (const char , mode_t, dev_t);
    int (mkdir) (const char , mode_t);
    int (unlink) (const char );
    int (rmdir) (const char );
    int (symlink) (const char , const char );
    int (rename) (const char , const char );
    int (link) (const char , const char );
    int (chmod) (const char , mode_t);
    int (chown) (const char , uid_t, gid_t);
    int (truncate) (const char , off_t);
    int (utime) (const char , struct utimbuf );
    int (open) (const char , struct fuse_file_info );
    int (read) (const char , char , size_t, off_t, struct fuse_file_info );
    int (write) (const char , const char , size_t, off_t,struct fuse_file_info );
    int (statfs) (const char , struct statfs );
    int (flush) (const char , struct fuse_file_info );
    int (release) (const char , struct fuse_file_info );
    int (fsync) (const char , int, struct fuse_file_info );
    int (setxattr) (const char , const char , const char , size_t, int);
    int (getxattr) (const char , const char , char , size_t);
    int (listxattr) (const char , char , size_t);
    int (removexattr) (const char , const char *);
};

Potentially with explanation from https://developer.ibm.com/articles/l-fuse/

And how to implement them

\section{Modularity}

Explain how to expand the VFS further

How are the modules using the core operations...