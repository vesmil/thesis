\chapter{Implementation}
\label{chap:implementation}

\section{FUSE Operations}\label{sec:fuse-ops}

Some valuable resources for learning about and working with FUSE include an IBM Developer article~\cite{ibm_fuse}, a tutorial from Facile Engineering~\cite{facile_fuse}, and a guide to writing a simple filesystem using FUSE~\cite{maastaar_fuse}.

To successfully implement the custom Virtual File System (VFS), various FUSE operations are required.
Some of the essential operations include getattr, readdir, open, read, write, mkdir, and rmdir, among others.
These operations are responsible for handling basic file system functionalities.

A brief overview of some crucial FUSE operations:

\begin{itemize}
    \item \textbf{getattr}: Retrieves the file or directory attributes, such as size, access permissions, and timestamps.
    \item \textbf{readdir}: Lists the contents of a directory.
    \item \textbf{open}: Opens a file for reading or writing.
    \item \textbf{read}: Reads data from an open file.
    \item \textbf{write}: Writes data to an open file.
    \item \textbf{mkdir}: Creates a new directory.
    \item \textbf{rmdir}: Deletes an existing directory.
    ...
\end{itemize}

To implement these FUSE operations, each operation's function must be defined in the custom VFS.
These functions will be responsible for performing the corresponding file system tasks and interacting with the underlying data structures.

\xxx{Some pseudo code...}


\section{Modularity}\label{sec:modularity}

One of the primary goals in designing the custom VFS is to ensure modularity, enabling easy expansion and the addition of new features or modules.
This modular approach allows developers to build upon the existing VFS without making extensive changes to the core implementation.

To achieve modularity, the VFS should be designed with well-defined interfaces and organized in a way that promotes separation of concerns.
This can be accomplished by organizing the code into distinct components, each responsible for specific tasks or features, such as encryption, versioning, or other desired functionalities.