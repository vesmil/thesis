\chapter{Preparation}
\label{chap:preparation}

With the architecture and design decisions established, the focus now shifts to the implementation of the custom and extendable VFS\@.
This concise chapter will outline the tools and technologies employed throughout the development process, ensuring a solid foundation for the implementation phase.


\section{Cross-platform build System}\label{sec:build-system-and-cross-platform-challenges}

CMake~\cite{cmake}, an open-source build system, was selected for this project due to its cross-platform compatibility, user-friendliness, and extensive support for various platforms.
CMake generates native build environments, such as Makefiles or project files for integrated development environments (IDEs), streamlining the development process.

\subsection{Platform-Specific Challenges}\label{subsec:platform-specific-challenges}

Despite the cross-platform advantages provided by CMake, several platform-specific challenges arose during development.
For instance, running applications on an M1 Mac proved difficult due to the lack of compatible binaries for the hardware.
While building the applications from source was an option, there was no compelling reason to do so for the purposes of this thesis.
Furthermore, the osxfuse port had not received an update in five years, exacerbating the situation.
On Windows, WinFsp~\cite{winfsp} is required to ensure FUSE compatibility, introducing additional hurdles for seamless cross-platform development.


\section{Gitlab CI}\label{sec:gitlab-ci}

The MFF faculty GitLab instance was utilized for this project\cite{gitlab_mff}.
GitLab CI/CD pipelines were employed to automate various tasks, such as checking code formatting with clang-format, building the project, running Google Test unit tests, and providing the ability to download build artifacts.
This approach facilitated a more efficient and organized development process.


\section{Docker}\label{sec:docker}

Docker\cite{docker} was incorporated into the project for several reasons.
First, it allowed for the creation of a consistent and reproducible development environment, ensuring that the project could be built and run across different platforms without issues.
Additionally, Docker provided two methods for running the VFS: one that displays debug output in a terminal, and another that runs the VFS in the background.
By integrating Docker into the project, the development and deployment process was simplified, allowing for a more efficient and effective implementation of the custom and extendable VFS.

\xxx{Provide example commands?}