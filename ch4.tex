\chapter{Implementation}
\label{chap:implementation}

\section{Essential FUSE Operations}\label{sec:fuse-ops}

To successfully implement a custom Virtual File System (VFS) using FUSE, it is essential to define and integrate various FUSE operations.
This section provides a brief overview of some essential FUSE operations, incorporating information from the Facile Engineering tutorial and IBM Developer article~\cite{ibm_fuse, facile_fuse}:

\begin{itemize}
    \item \textbf{getattr}: Retrieves the metadata of a given path.
    This operation is always called before any other operation made on the filesystem.
    It is responsible for reading the file or directory attributes, such as size, access permissions, and timestamps.
    \item \textbf{readdir}: Lists the contents of a directory, filling a buffer with the structure of the accessed directory.
    \item \textbf{open}: Called when the system requests a file to be opened.
    \item \textbf{read}: Called when FUSE is reading data from an opened file.
    It should return exactly the number of bytes requested and fill the buffer with the content of those bytes.
    \item \textbf{write}: Writes data to an open file.
    \item \textbf{mkdir}: Creates a new directory.
    \item \textbf{rmdir}: Deletes an existing directory.
    \item \textbf{rename}: Renames a file or directory.
    \item \textbf{truncate}: Changes the size of a file.
\end{itemize}

Besides these operations a full-featured filesystem might need operations such as \texttt{mknod}, \texttt{unlink}, \texttt{symlink}, \texttt{link}, \texttt{chmod}, \texttt{chown}, \texttt{utime}, \texttt{statfs}, \texttt{flush}, \texttt{release}, \texttt{fsync}, \texttt{setxattr}, \texttt{getxattr}, \texttt{listxattr}, and \texttt{removexattr}.

To create a filesystem with FUSE, a structure variable of type fuse\_operations should be declared and passed to the fuse\_main function.
The fuse\_operations structure contains pointers to functions that will be called when the appropriate action is required.
This is in my implementation done in the \texttt{FuseWrapper} class and all the other classes simply implement the required functions.

\section{Encryption}

What is best way to encrypt a file with user provided password?

Using AES\ldots from Crypto++

Some code snippet that just shows wrapping read and write operations.

\ldots

\section{Versioning}

I have decided to not do diff but just store the whole file in the versioning system.

Also just show how wraping read and write operations works.

Also show the hook in read that \ldots