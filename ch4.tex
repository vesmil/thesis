\chapter{Implementation}
\label{chap:implementation}

\section{Essential FUSE Operations}\label{sec:fuse-ops}

To successfully implement a custom Virtual File System (VFS) using FUSE, it is essential to define and integrate various FUSE operations.
This section provides a brief overview of some essential FUSE operations, incorporating information from the Facile Engineering tutorial and IBM Developer article~\cite{ibm_fuse, facile_fuse}:

\begin{itemize}
    \item \textbf{getattr}: Retrieves the metadata of a given path.
    This operation is always called before any other operation made on the filesystem.
    It is responsible for reading the file or directory attributes, such as size, access permissions, and timestamps.
    \item \textbf{readdir}: Lists the contents of a directory, filling a buffer with the structure of the accessed directory.
    \item \textbf{open}: Called when the system requests a file to be opened.
    \item \textbf{read}: Called when FUSE is reading data from an opened file.
    It should return exactly the number of bytes requested and fill the buffer with the content of those bytes.
    \item \textbf{write}: Writes data to an open file.
    \item \textbf{mkdir}: Creates a new directory.
    \item \textbf{rmdir}: Deletes an existing directory.
    \item \textbf{rename}: Renames a file or directory.
    \item \textbf{truncate}: Changes the size of a file.
\end{itemize}

Besides these operations a full-featured filesystem might need operations such as \texttt{mknod}, \texttt{unlink}, \texttt{symlink}, \texttt{link}, \texttt{chmod}, \texttt{chown}, \texttt{utime}, \texttt{statfs}, \texttt{flush}, \texttt{release}, \texttt{fsync}, \texttt{setxattr}, \texttt{getxattr}, \texttt{listxattr}, and \texttt{removexattr}.

To create a filesystem with FUSE, a structure variable of type fuse\_operations should be declared and passed to the fuse\_main function.
The fuse\_operations structure contains pointers to functions that will be called when the appropriate action is required.
This is in my implementation done in the \texttt{FuseWrapper} class and all the other classes simply implement the required functions.

\section{Encryption}\label{sec:encryption}

I will use a Crypto++ to handle the encryption and decryption of files within the custom VFS\@.
In order to use an user-provided password, Advanced Encryption Standard (AES) algorithm will be utilized.
AES is a widely-used symmetric encryption algorithm that provides a good balance between security and performance.
Alternatives include other symmetric encryption algorithms like DES (Data Encryption Standard) and 3DES (Triple DES), but they are slower and less secure compared to AES.

To integrate the encryption, I have modified the existing read and write operations to include the encryption and decryption processes.
The following code snippet demonstrates how to wrap the read operation with encryption.

\begin{lstlisting}[language=c++, basicstyle=\ttfamily\small]
int EncryptionVfs::read(const std::string &file_path,
                     char *buffer,
                     size_t size,
                     off_t offset) {
    wrapped_vfs_->read(file_path, buffer, size, offset);
    decrypt(buffer, size, password);
}
\end{lstlisting}

\section{Versioning}\label{sec:versioning2}

Instead of using a diff-based approach for versioning, I have chosen to store the entire file in the versioning system as it simplifies the implementation.
Diff-based approach could be implemented in the future.

Similar to the encryption implementation, the read and write operations were wrapped to include versioning.
The following code snippet shows high-level overview of how to wrap the write operation with versioning functionality.

\begin{lstlisting}[language=c++, basicstyle=\ttfamily\small]
int VersioningVfs::write(const std::string &file_path,
                     char *buffer,
                     size_t size,
                     off_t offset) {
    char* prev_buffer;
    read_prev_version(file_path, prev_buffer);

    if (prev_buffer != buffer) {
        // This will store an invisible copy of the file
        store_version(file_path, buffer, size);
    }

    wrapped_vfs_->write(file_path, buffer, size, offset);
}
\end{lstlisting}