\chapter{Preparation}
\label{chap:preparation}

With architecture and design decisions made, it is time to start implementing the VFS\@.
In this rather brief chapter, I will discuss the tools and technologies used in the development process.

\xxx{...}

\section{Cross-platform build System}\label{sec:build-system-and-cross-platform-challenges}

CMake~\cite{cmake} is an open-source build system.
It was chosen as the build system for this project due to its cross-platform compatibility, ease of use, and broad support for various platforms.
It generates native build environments, such as Makefiles or project files for integrated development environments (IDEs).

\subsection{Platform-Specific Challenges}\label{subsec:platform-specific-challenges}

Despite CMake's cross-platform capabilities, certain platform-specific challenges were encountered during the development process.
When attempting to run applications on an M1 Mac, I encountered an issue where there were no compatible binaries available for the hardware.
While I could have built the applications from source, I did not have a specific reason to do so for my thesis.
Additionally, the osxfuse port had not been updated for five years, further complicating matters.
On Windows, the use of WinFsp~\cite{winfsp} is necessary to provide FUSE compatibility, presenting additional challenges for seamless cross-platform development.

\section{Gitlab CI}\label{sec:gitlab-ci}

\xxx{...}

\section{Docker}\label{sec:docker}

\xxx{...}