\chapter{Analysis}
\label{chap:analysis}

\section{Analysis of Alternatives}\label{sec:alternatives}

In order to judge the potential benefits of the proposed virtual file system, it is essential to examine alternative solutions and compare their capabilities, limitations, and applicability.
This section will analyze existing non-virtual and virtual file systems with versioning and encryption features, as well as higher-level applications offering similar functionality.

Non-virtual versioning file systems is a well-documented concept, and there are several implementations available.
Notable non-virtual versioning file systems include OpenVMS, a versioning file system from Digital Equipment Corporation that automatically creates new instances of files with version numbers appended, and NILFS, a Linux-based log-structured file system that supports versioning and continuous snapshotting of the entire file system.

Furthermore, several virtual file systems (VFS) with versioning capabilities have been developed:

\begin{itemize}
    \item User space file systems implemented with FUSE:
    \begin{itemize}
        \item A simple versioning file system for Linux using FUSE~\cite{simple_vfs}, written in Go.
        \item Wayback: A User-level Versioning File System for Linux~\cite{wayback_vfs}, developed in Perl for the USENIX 2004 Annual Technical Conference using an older version of FUSE\@.
    \end{itemize}
    \item Copy-on-Write Version Support for VFS under Linux by Stephan Müller and Sven Widmer~\cite{vvfs}, implemented as a kernel patch.
    \item A versioning virtual filesystem by Steve Huntley~\cite{huntley_vvfs}, written in Tcl.
    However, this solution primarily serves as a language demonstration rather than a practical implementation.
\end{itemize}

These existing solutions have various constraints, such as being platform-specific (mainly Linux-only) or discontinued.
While virtual file systems with encryption are basically nonexistent, there are non-virtual alternatives such as the Encrypted File System (EFS).
EFS provides cryptographic protection of individual files on NTFS file system volumes using a public-key system, enhancing the security of sensitive data, especially on lost or stolen devices.

The closest example to the desired functionality is rvault~\cite{rvault}, which focuses on encrypting small files (passwords, keys, and secrets) and makes them accessible through one-time password authentication.
However, this does differ from the intended functionality of the proposed VFS\@.

The main competition thus stems from higher-level applications that provide versioning and encryption features.
These applications have their own set of limitations, such as requiring a constantly running background program with access to all files and offering limited extensibility for incorporating other features.
In most instances, integrating additional functionalities into these applications would not be practical or feasible.

Despite the limitations of existing solutions, it is worth noting that some operating systems, specifically macOS, provide built-in support for versioning and encryption features.
MacOS's Time Machine offers versioning capabilities, allowing users to revert to previous versions of files or restore deleted data.
Furthermore, macOS incorporates FileVault, a native full-disk encryption solution that secures data.
However, these features are not available on other platforms, and they may not provide the desired level of control over the process.

\section{FUSE: Rationale and Alternatives}\label{sec:fuse-analysis}

In the development of a virtual file system, there are two primary approaches to consider: writing kernel drivers or utilizing a user space library.
Writing kernel drivers provides low-level access to the operating system, but it requires extensive knowledge of the kernel, as well as platform-specific implementations.
This method can be time-consuming, error-prone, and challenging to maintain.

An alternative to writing kernel drivers is a library such as using FUSE, a popular tool for creating virtual file systems in user space without modifying kernel code.
FUSE offers a comprehensive API for defining file system operations, making it an appropriate choice for building a VFS\@.
It allows developers to focus on the functionality and logic of their custom file system, rather than the intricate details of kernel programming.
Yet, FUSE is not without its limitations.
It is arguable whether it is the best choice for a C++ implementation, as it is written in C, but more on that later.

Although FUSE was initially designed for Linux, variants are available for other platforms, ensuring cross-platform compatibility:

\begin{itemize}
    \item \textbf{Linux}: libfuse - The reference implementation of FUSE~\cite{libfuse}.
    \item \textbf{macOS}: FUSE for macOS - A macOS port of FUSE~\cite{osxfuse}.
    \item \textbf{Windows}: WinFsp - A Windows File System Proxy that provides FUSE-compatible functionality~\cite{winfsp}.
\end{itemize}

Considering the advantages of user space development and the availability of FUSE for multiple platforms, FUSE is my preferred choice for implementing the proposed VFS\@.
This approach enables the development of a cross-platform VFS with versioning and encryption capabilities while avoiding the complexity of kernel driver development.

\section{Other libraries}\label{sec:other-libraries-analysis}

Altough FUSE serves as the foundation for the VFS implementation, several other libraries will be employed to address various aspects of the project.

\subsection{Encryption}\label{subsec:encryption-analysis}

Crypto++~\cite{crypto_pp} is an extensive, open-source C++ class library that offers a broad range of cryptographic schemes, encompassing encryption, hashing, and authentication algorithms.
By integrating Crypto++ into the custom VFS, file-level encryption can be achieved, ensuring the security and privacy of the data stored within.


\subsection{Testing}\label{subsec:gtest}

Google Test~\cite{google_test}, commonly referred to as gtest, is a widely-used and adaptable C++ testing framework developed by Google.
This library enables efficient testing of individual components and the overall functionality of the custom VFS\@.
By employing gtest, the quality, reliability, and performance of the custom VFS can be evaluated, guaranteeing that it fulfills the requirements and expectations outlined in this thesis.

\section{Cross-platform build System}\label{sec:build-system-and-cross-platform-challenges}

CMake~\cite{cmake} is an open-source build system.
It was chosen was chosen as the build system for this project due to its cross-platform compatibility, ease of use, and broad support for various platforms.
It generates native build environments, such as Makefiles or project files for integrated development environments (IDEs).

\subsection{Platform-Specific Challenges}\label{subsec:platform-specific-challenges}

Despite CMake's cross-platform capabilities, certain platform-specific challenges were encountered during the development process.
When attempting to run applications on an M1 Mac, I encountered an issue where there were no compatible binaries available for the hardware.
While I could have built the applications from source, I did not have a specific reason to do so for my thesis.
Additionally, the osxfuse port had not been updated for five years, further complicating matters.
On Windows, the use of WinFsp~\cite{winfsp} was necessary to provide FUSE compatibility, presenting additional challenges for seamless cross-platform development.