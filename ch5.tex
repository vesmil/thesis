\chapter{Implementation}
\label{chap:implementation}

Once the preparation phase was complete, it was time to begin the implementation itself.


\section{Essential FUSE Operations}\label{sec:fuse-ops}

It was first essential to define and integrate various FUSE operations.
This section provides a brief overview of some of the essential operations, incorporating information from the Facile Engineering tutorial and IBM Developer article~\cite{ibm_fuse, facile_fuse}:

\begin{itemize}
    \item \textbf{getattr}: Retrieves the metadata of a given path.
    This operation is always called before any other operation made on the filesystem.
    It is responsible for reading the file or directory attributes, such as size, access permissions, and timestamps.
    \item \textbf{readdir}: Lists the contents of a directory, filling a buffer with the structure of the accessed directory.
    \item \textbf{mknod}: Creates a non-directory node, such as a file.
    \item \textbf{unlink}: Deletes a file.
    \item \textbf{open}: Called when the system requests a file to be opened.
    \item \textbf{read}: Called when FUSE is reading data from an opened file, while filling the buffer with the content of those bytes.
    \item \textbf{write}: Writes data to an open file.
    \item \textbf{mkdir}: Creates a new directory.
    \item \textbf{rmdir}: Deletes an existing directory.
    \item \textbf{rename}: Renames a file or directory.
\end{itemize}

Besides these operations a full-featured filesystem might need operations such as \texttt{truncate}, \texttt{symlink}, \texttt{link}, \texttt{chmod}, \texttt{chown}, \texttt{utime}, \texttt{statfs}, \texttt{flush}, \texttt{release}, \texttt{fsync}, \texttt{setxattr}, \texttt{getxattr}, \texttt{listxattr}, and \texttt{removexattr}.

To create a filesystem with FUSE, a structure variable of type fuse\_operations should be declared and passed to the fuse\_main function.
The fuse\_operations structure contains pointers to functions that will be called when the appropriate action is required.
This is, as described earlier\ref{sec:fuse-in-cpp}, done in the \texttt{FuseWrapper} class and all the other classes simply implement the required functions.

\subsection{Operations for the Base VFS}\label{subsec:base-ops}

The core of the implementation is provided by \texttt{CustomVfs} class, which is responsible for handling all file system operations.
This particular class mainly redirects the operations to the appropriate system calls, which are then handled by the operating system.
The exact way this is done is using a backing directory, which is mounted to the \texttt{.customvfs-{mount}} directory that is created when the filesystem is mounted.
This particular directory is then used to store all the files and directories that are created by the user by simply forwarding the operations to the backing directory.
Consequently, the debugging process is simplified as the user can easily access the backing directory and verify that the operations are being handled correctly.

Using the base VFS is also then as simple as creating an object and invoking the \texttt{main} method, as demonstrated in the following code snippet.\ref{lst:main}

\begin{lstlisting}[language=c++, basicstyle=\ttfamily\small, caption={Main method of the \texttt{CustomVfs} class.}, label={lst:main}]
int main(int argc, char* argv[]) {
    CustomVfs custom_vfs();
    custom_vfs.main(argc, argv);
    return 0;
}
\end{lstlisting}

The arguments passed to the \texttt{main} method are then simply forwarded to the wrapper and consequently to the FUSE library.
This allows the user to specify the mount point and other options such as the debug mode or ignoring non-empty mount point.


\section{Encryption}\label{sec:encryption}

To implement the prototype module of encryption, I have chosen to use the XChaCha20-Poly1305 as described in the initial chapter~\ref{subsec:xchacha20-poly1305}.
I particularly used this algorithm implemented by the libsodium library\ref{subsec:encryption-analysis}.

There are two main approaches to use the added encryption functionality.
The first approach is to use a user-defined password, which is then employed to generate the encryption key.
This method is not very practical, as the user has to manually lock and unlock files, since there is no mechanism to request the password when accessing a file.

The second approach involves using a key file.
This method is more convenient, as users can store the key file in a secure location and access their files without additional steps.
However, this brings up the question of when to encrypt and decrypt the files.

Encryption and decryption of files are easily performed upon opening and closing the file, respectively.
In contrast, handling directories is more complicated, as there is no way to identify when the directory is being accessed.
As a result, I have chosen to temporarily decrypt file names when reading the directory, while individual file contents are decrypted only when the file is opened.
This approach reduces unnecessary encryption and decryption operations and maintains relatively fast performance.

The encryption itself was done by asking the parent vfs for the proper file path and then using the libsodium library to encrypt or decrypt the file.
Encryption of directories was then done by simply encrypting the files and their names in the directory recursively.

\section{Versioning}\label{sec:versioning2}

The versioning module prototype is as of now rather limited since implementing a full-featured versioning system would be a significant undertaking.
Namely, instead of using a diff-based approach for versioning, I have chosen to store the entire file in the versioning system as it simplifies the implementation.
Diff-based approach could be then implemented in the future.

Similar to the encryption implementation, the read and write operations were wrapped to include versioning.
The following code snippet shows a high-level overview of how to wrap the write operation with versioning functionality.

\begin{lstlisting}[language=c++, basicstyle=\ttfamily\small]
int VersVfs::write(const std::string &file_path,
                   char *buf, size_t size, off_t off) {
    store_previous_version(file_path);
    if (off > 0) {
        prepare_newfile_for_write(file_path);
    }

    wrapped_vfs_->write(file_path, buf, size, off);
}
\end{lstlisting}

Meaning, the current version is always stored in a file with the real name, whereas the previous versions are suffixed with special symbol and their version number.
With this approach operations such as restore, delete, or list are relatively straightforward to implement.
Restore just creates a new stored version with the current content of the file and renames the restored version to the real name.
Delete operation then simply removes the file and all its versions.
And listing just finds all the files with the real name and the special symbol and returns them.