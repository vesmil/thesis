\chapter{Evaluation}\label{chap:evaluation}

Once the implementation of the custom VFS is complete, it is necessary to evaluate the project.
The evaluation is performed in terms of its functionality, usability, reliability, security, and performance.


\section{Usability}\label{sec:usability}

The VFS itself is user-friendly, as the user only needs to specify the mount directory on the command line to start using it.
The features are currently accessed through a set of CLI tools, which are slightly less user-friendly, but that could be improved over time.
Despite the limited effort put into developing the CLI, it remains a functional and accessible interface.

In future iterations, the usability of the custom VFS could be further enhanced by providing a graphical user interface (GUI) to make it even more accessible for users who are not familiar with command-line interfaces.
Moreover, more detailed documentation and tutorials could be provided to facilitate user onboarding and promote the adoption of the custom VFS\@.


\section{Reliability and Security}\label{sec:reliability-and-security}

The VFS implementation has undergone testing through unit tests using Google Test, ensuring good code coverage and the core features, such as encryption and communication with kernel code, rely on well-tested external libraries.

However, there is still room for improvement in the core VFS implementation to further enhance its reliability.
In future iterations, a more extensive suite of tests could be developed, covering not only unit tests but also integration and stress tests.
This would help ensure that the VFS behaves correctly under various scenarios and is resilient against potential failures.

In terms of security, the implemented encryption relies on the Libsodium library, which is a well-tested and widely used library for encryption.


\section{Performance}\label{sec:performance}

Performance was not the primary focus of this project, as it is still a prototype.
For that reason, there were no performance tests conducted.
On the other hand, there are no known performance bottlenecks in the current implementation, as the VFS is built on top of the FUSE library, which is known for its high performance.

Future work could involve conducting performance tests to identify potential areas of improvement and optimize the VFS accordingly.
This would be particularly important for applications with high I/O demands or large-scale deployments.


\section{Feature overview}\label{sec:feature-overview}

To showcase the features of the custom VFS; the following table compares created VFS to similar software.
I have even included some file systems, as they have some of the same features.

It is important to note that the custom VFS is not a direct competitor to the other software, as it is a prototype and not a fully-fledged product.
Besides that, the custom VFS has the ability to combine features easily, which could be used to create a more complete product.

Hopefully, most of the features are self-explanatory, but there is a brief explanation of the more ambiguous ones just in case.
The versioning \textit{Type} means whether it stores full snapshots or only the differences between snapshots,
the \textit{Auto} in versioning means that the VFS stores snapshots automatically, without any user intervention, whereas
\textit{Auto unlock} in encryption describes whether the files are automatically unlocked on access.
The \textit{Dir} and \textit{File} columns simply tell if the VFS can encrypt directories and files.

\clearpage

\renewcommand{\checkmark}{\tikz\fill[scale=0.4](0,.35) -- (.25,0) -- (1,.7) -- (.25,.15) -- cycle;}

\begin{table}[ht]
    \centering
    \caption{Versioning Tools Evaluation}
    \begin{tabular}{|c|c|c|c|c|c|c|c|}
        \hline
        & \textbf{Win} & \textbf{Linux} & \textbf{macOS} & \textbf{Files} & \textbf{Dirs} & \textbf{Type} & \textbf{Auto} \\
        \hline
        This VFS   & --           & \checkmark     & \checkmark     & \checkmark     & --            & Full          & \checkmark    \\
        \hline
        Btrfs        & --           & \checkmark     & --             & \checkmark     & \checkmark    & Diff          & \checkmark    \\
        \hline
        ZFS          & --           & \checkmark     & \checkmark     & \checkmark     & \checkmark    & Diff          & \checkmark    \\
        \hline
        Time Machine & --           & --             & \checkmark     & \checkmark     & \checkmark    & Diff          & \checkmark    \\
        \hline
        Shadow Copy  & \checkmark   & --             & --             & \checkmark     & \checkmark    & Full          & \checkmark    \\
        \hline
        Git          & \checkmark   & \checkmark     & \checkmark     & \checkmark     & \checkmark    & Diff          & --            \\
        \hline
    \end{tabular}
    \label{tab:versioning-evaluation}
\end{table}

\vspace{2em}

\begin{table}[ht]
    \centering
    \caption{Encryption Tools Evaluation}
    \begin{tabular}{|c|c|c|c|c|c|c|}
        \hline
        & \textbf{Win} & \textbf{Linux} & \textbf{macOS} & \textbf{Password} & \textbf{Key} & \textbf{Auto Unlock} \\
        \hline
        This VFS  & \texttimes   & \checkmark     & \checkmark     & \checkmark        & \checkmark   & \checkmark           \\
        \hline
        VeraCrypt & \checkmark   & \checkmark     & \checkmark     & \checkmark        & \checkmark   & \texttimes           \\
        \hline
        BitLocker & \checkmark   & \texttimes     & \texttimes     & \checkmark        & \checkmark   & \checkmark           \\
        \hline
        AxCrypt   & \checkmark   & \texttimes     & \texttimes     & \checkmark        & \texttimes   & \checkmark           \\
        \hline
        7-Zip     & \checkmark   & \checkmark     & \texttimes     & \checkmark        & \texttimes   & \texttimes           \\
        \hline
    \end{tabular}
    \label{tab:encryption-evaluation}
\end{table}


