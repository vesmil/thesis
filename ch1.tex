\chapter{Theoretical overview}
\label{chap:refs}

The development of a custom Virtual File System (VFS) with versioning and encryption capabilities requires an understanding of several key concepts and technologies.
This chapter aims to provide the necessary background and context to deeper understand the VFS implementation discussed in later chapters.
Even for those already familiar with the subject, a brief refresher might be helpful.


\section{Understanding File Systems}\label{sec:file-systems}

Let's start with what a file system is and what it does, as it is important for understanding the VFS itself.

A file system is a critical component of any operating system, tasked with managing and organizing data stored on a device.
As described by Oja et al.~\cite{oja-fs}, a filesystem represents the methods and data structures used by an operating system to maintain files on a disk or partition, that is, how files are organized on the disk.
This consists of multiple operations such as creation, reading, modifying, and deleting files or directories.
This is done in order to allow users and applications to interact seamlessly with the underlying storage medium.

It is important to distinguish between a disk or partition and the file system it contains.
While some programs operate directly on raw disk sectors or partitions, which could potentially damage or corrupt existing file systems, most programs interact with a file system.

There are various types of file systems, each tailored for specific operating systems and purposes.
Notable file systems include NTFS (New Technology File System) for Windows, HFS+ for macOS, and ext4 (fourth extended filesystem) for Linux.
Cross-platform compatibility can be attained using universal file systems like FAT32 or exFAT, which are compatible with multiple operating systems.

File systems employ diverse techniques to organize data, such as partitioning, allocation units, and indexing.
Additionally, file systems manage metadata, which is crucial for organizing and accessing stored data.
Metadata encompasses information such as file names, creation and modification timestamps, file sizes, and access permissions.


\section{Virtual File Systems}\label{sec:virtual-file-systems}

A Virtual File System (VFS) is an essential abstraction layer in modern operating systems, designed to ease the interaction between various file systems and user applications.
The VFS functions as an intermediary, enabling applications to access various file systems through a unified interface, regardless of the underlying file system's specific characteristics or structure.

According to a book by David A. Rusling~\cite{rusling-linux}, the VFS manages various mounted file systems by maintaining data structures that describe the entire virtual file system and the real mounted file systems.
The VFS uses superblocks and inodes, similar to the EXT2 file system, to describe files and directories within the system.
Each file system register with the VFS during operating system initialization.
File system modules are loaded as needed, allowing VFS to read their superblocks.
The VFS maintains a list of mounted file systems and their VFS superblocks, with each superblock containing pointers to specific file system functions.

The VFS inodes are traversed as the system's processes access directories and files.
Frequently accessed inodes are kept in the inode cache for quicker access.
Additionally, Linux VFS uses a common buffer cache, independent of the file systems, to cache data buffers from underlying devices.

VFS also plays a critical role in operating system design as it not only offers file system independence, but also provides benefits such as improved security or better performance.
In this thesis, I will also demonstrate that the extensibility of the VFS is a valuable advantage, enabling the incorporation of advanced features without requiring kernel modifications.

\subsection{VFS Operations}\label{subsec:vfs-operations}

The Virtual File System (VFS) provides a range of operations to interact with files and directories.
These operations are implemented as system calls, such as \texttt{open(2)}, \texttt{stat(2)}, \texttt{read(2)}, \texttt{write(2)}, and \texttt{chmod(2)}, which are called from a process context.

The way this is implemented in Linux VFS is described by an article by Richard Gooch and Pekka Enberg~\cite{vfs}.
During a system call, the VFS translates a pathname into a directory entry (dentry) by searching through the dentry cache.
However, if the cache is too small to fit all dentries in RAM, the VFS may have to create dentries and load inodes to resolve the pathname.
Each dentry typically has a pointer to an inode, which is a file system object that may be on disk or in memory.

When a file is opened, a file structure is allocated and initialized with a pointer to the dentry and file operation member functions taken from the inode data.
The \texttt{open()} file method is then called to enable the file system implementation to perform its work.
Finally, the file structure is placed into the file descriptor table for the process.

To read, write, and close files, the user-space file descriptor is used to call the appropriate file structure method.
As long as the file is open, the corresponding dentry and VFS inode remain in use.


\section{Encryption}\label{sec:encryption-approaches}

Another concept that has to be briefly discussed is encryption. It is simply a process of applying cryptographic algorithms (encoding) to information in such a way that only authorized parties can access it.
In this thesis, I will mainly use encryption as a means to transform the contents of files and directories into an unreadable format, which can only be deciphered with the appropriate decryption key. Utilizing encryption enables users to effectively safeguard their data against unauthorized access, data breaches, and other malicious activities.

In file and directory encryption, symmetric key algorithms are employed due to their computational efficiency and robust security properties. While the Advanced Encryption Standard (AES) is a widely used option, I have decided to use the more modern and faster XChaCha20-Poly1305 algorithm for this thesis. To generate the encryption key from a user-provided password, password-based key derivation functions can be leveraged.

\subsection{XChaCha20-Poly1305}\label{subsec:xchacha20-poly1305}

XChaCha20-Poly1305 is a symmetric encryption algorithm that combines the XChaCha20 stream cipher with the Poly1305 message authentication code (MAC) to provide both confidentiality and integrity. This modern encryption algorithm is considered faster than AES, especially in software implementations, making it a suitable choice for this thesis.

The algorithm uses a 256-bit key and a 192-bit nonce. The larger nonce size in XChaCha20-Poly1305 reduces the risk of nonce reuse, which can compromise security. To encrypt data using XChaCha20-Poly1305, a user first generates a secret key, which is a random sequence of bits with a length of 256 bits. Then, the algorithm uses the secret key to transform the plaintext into ciphertext. The ciphertext can only be decrypted back to its original plaintext form using the same secret key that was used to encrypt it.

Like with AES, a password-based key derivation function (PBKDF) can be utilized to derive an encryption key from a password. This enables users to access their encrypted data using a memorizable password rather than having to manage and store a long, randomly generated encryption key.

\section{Versioning}\label{sec:versioning}

Versioning is simply a technique used to maintain multiple copies or states of a file or directory, allowing users to access and restore previous versions as needed.
It enables tracking of changes made to files over time, facilitating collaboration and recovery from accidental data loss or corruption.

Usually, versioning is implemented by storing multiple copies of a file or directory, each representing a different version.
This approach is straightforward and easy to implement, but it can be inefficient in terms of storage space and performance.
To address these issues, versioning can be implemented using a snapshot-based approach, which stores only the differences between versions, rather than storing multiple copies of the same file.