\chapwithtoc{Conclusion and Future Work}

This thesis has successfully presented the design and implementation of a modular and easily extensible VFS in user space.
Moreover, I have provided prototypes for encryption and versioning modules, which seamlessly integrate with the core functionality and require minimal user effort.
The proposed VFS prototypes enable users to create snapshots for potential rollbacks and effortlessly encrypt individual files or entire directories using a password or key, offering temporary decryption when necessary.
Importantly, the VFS transcends the limitations of similar programs, as it can be mounted on any file system, including network file systems, and its features can be easily layered on top of one another.

As we look ahead, there are several opportunities for future development and research to enhance the results of this thesis:

\begin{itemize}
    \item \textbf{Optimization}: The current VFS implementation can be optimized for improved performance, encompassing both storage efficiency and processing speed.
    \item \textbf{GUI Integration}: Integrating the VFS features into a graphical user interface or even existing file managers will provide users with a more intuitive and visually appealing experience.
    \item \textbf{Additional Features}: Developers can create and incorporate new features or modules into the VFS, such as versioning based on diff, compression, deduplication, or support for various storage backends (e.g., cloud storage or distributed file systems).
    \item \textbf{Real-world Deployment}: Refining and testing the VFS for real-world deployment will enable the evaluation of its performance, reliability, and usability in more diverse and demanding environments.
    \item \textbf{Windows Support}: The VFS can be ported to Windows, allowing users to access its features on a wider range of operating systems.
\end{itemize}

In conclusion, this thesis has substantiated the feasibility of creating an extensible Virtual File System with support for encryption and versioning.
Future work can expand upon this foundation, developing a more robust, efficient, and feature-rich VFS suitable for a wide range of applications.