
\chapwithtoc{Introduction}

%\item What is the nature of the problem the thesis is addressing?
With the rising importance of data, many users would benefit from advanced features such as versioning and file-level
encryption to their data storage without the need for specialized software.
In this thesis, I aim to provide a solution to this problem by introducing a virtual file system (VFS) that allows users
to add these features to their existing file systems with ease.
Specifically, the user will be able to create snapshots to later rollback and will be able to encrypt specific files or
even directories with password and then temporarily decrypt them in a convenient way.

%\item What is the common approach for solving that problem now?
Usually if user wants mentioned features, he has to download two separate binaries that he will use.
As an example for versioning every reader probably knows \texttt{git}.
With that user gets the ability to create a so called repository with information about file changes stored in \texttt{.git} directory.
To password protect files, there is plenty of software to choose from and just to name any, one may use Folder Lock.

%\item How this thesis approaches the problem?
But that doesn't maximize the user's convenience as mentioned approaches come with limitation of having to use specialized
software to access the files and the software usually leaks the implementation details.
My thesis on the other hand, provides these features in form of virtual file system (VFS) - some form of intermediate layer
between OS and specific file systems.

%\item What can the reader expect in the individual chapters of the thesis?
The individual chapters of the thesis will provide an in-depth explanation what a virtual file system is and how it works,
how to create a custom multiplatform VFS, and the approach used to add versioning and file-level encryption to it.
Finally, the thesis will conclude with an evaluation of the VFS's performance, usability, and security.